%% Packages indispensables
\documentclass[a4paper, 12pt]{article}
\usepackage[utf8]{inputenc}
\usepackage[T1]{fontenc}
\usepackage[english]{babel}

\usepackage[a4paper]{geometry}
\usepackage{graphicx}

\newcommand{\package}[1]{\texttt{#1}}
\newcommand{\kv}[2]{\textit{<#1>}\texttt=\textit{<#2>}}
\newcommand{\key}[3]{\textbf{\texttt{#1} (default value: \texttt{#2})} #3}
\newcommand{\commande}[1]{\texttt{\textbackslash#1}}


\usepackage{hyperref}
\hypersetup{
	colorlinks=true,
	linkcolor=blue,
	urlcolor=blue,
}

\title{Package \texttt{playcards.sty} for \LaTeX}
\date{\today}
\author{Clément Pagès \texttt{contact -- arobase -- clementpages point fr}}

\begin{document}
\maketitle

This small package provides commands to draw playcards, with width 59 mm and height 89 mm, which are typicall cards dimensions.

\tableofcontents
\section*{Thanks}
Thanks to Christophe Poulain for his amazing \href{https://ctan.org/pkg/profcollege}{ProfCollege} package, which source code was useful to design syntax commands we use here. And more generally, thank you for this amazing package!

Also thanks to \href{https://tex.stackexchange.com/users/1948/didest}{didest} who asked a question on \href{https://tex.stackexchange.com/questions/47924/creating-playing-cards-using-tikz}{StackExchange}. I used it to build this package.

\section{Download, installation, requirements}
	\subsection{Automatically loaded packages}
Package available on CTAN : \url{https://ctan.org/pkg/playcards}. It contains one single file: \texttt{playcards.sty}.

Some packages are automatically loaded. They are installed by default on most configurations:
\begin{description}
	\item[\package{tikz}] and some libraries.
	\item[\package{simplekv}] to manage optional parameters with a \kv{key}{value} system.
	\item[\package{graphicx}] for pictures. 
	\item[\package{contour}] for text shadows and borders.
\end{description}


	\subsection{Recommended packages, not automatically loaded}
To draw cards one by one, there are nos constraints on margins an paper format.

To get a full page of identical cards, paper has to be A4 format and command \commande{usepackage[scale=0.95]\{geometry\}} is required in preamble.

\section{Generalities}
Commands are based on a key/value system, with model \texttt{\textbackslash command[\kv{key1}{value1}, \kv{key2}{value2}, …]\{param1\}\{param2\}}.

All keys are optional and if one does not fill them, they get a default value.

All length are given in millimeters.

\section{Provided commands}
	\subsection{Package options}
By default, font for text on cards is the same as document's. You can pass in option the parameter \texttt{anttor} so that cards font will be \href{https://tug.org/FontCatalogue/antykwatorunska/}{Antykwa Toruńska}.

	\subsection{\commande{drawcardsrecto} command}
This command has one required parameter : text written in the center of the card. It fills an A4 paper with 9 identical cards. This command only draws front side of the card. There is an example figure \ref{fig:recto}.
\begin{figure}[h]\begin{center}
	\caption{\commande{drawcardsrecto[trame=false]\{5\}}}
	\includegraphics{fig01.pdf}\label{fig:recto}
\end{center}\end{figure}

Optional parameters:
\begin{itemize}
	\item \key{borders}{true}{Removes borders.}
	\item \key{trame}{true}{If \texttt{true}, fills an A4 paper with cards. If \texttt{false}, draws one only card. To work properly, \texttt{trame=true} requires A4 paper and \commande{usepackage[scale=0.95]\{geometry\}} in preamble.}
	\item \key{corners}{true}{If \texttt{true}, card's contents is reproduced in corners. If \texttt{false}, it is not.}
	\item \key{backgroundImg}{true}{If \texttt{true}, prints background. If \texttt{false}, no background.}
	\item \key{backgroundColor}{red}{Specifies background color. Ignored if \texttt{backgroundImg=false}. Uses colors defined in the \href{https://www.ctan.org/pkg/xcolor}{xcolor} package.}
	\item \key{contentsFontSize}{120}{Font size (in pt) of text in card center.}
	\item \key{circleRay}{20}{Ray of the white circle the center of the card. Value \texttt 0 means no circle.}
\end{itemize}


	\subsection{\commande{drawcardsverso} command}
It is possible, but not required, to draw card back side. Front sides must be on one page and back sides on an other. You must have as many front sides as back sides. There is an example figure \ref{fig:verso}.

To get a correct alignment, place commands in a  \verb!\begin{center}...\end{center}! environment.
\begin{figure}[h]\begin{center}
	\caption{\commande{drawcardsverso[trame=false,contentsFontSize=40]\{Exemple\}}}
	\includegraphics{fig02.pdf}\label{fig:verso}
\end{center}\end{figure}

Optional parameters:
\begin{itemize}
	\item \key{backgroundImg}{true}{If \texttt{true}, prints background. If \texttt{false}, no background.}
	\item \key{trame}{true}{If \texttt{true}, fills an A4 paper with cards. If \texttt{false}, draws one only card. To work properly, \texttt{trame=true} requires A4 paper and \commande{usepackage[scale=0.95]\{geometry\}} in preamble.}
	\item \key{contentsFontSize}{120}{Font size (int pt) of text in card center.}
\end{itemize}

\end{document}
